\documentclass[oneside,11pt,nocover,
  printed, %% This option enables the default options for the
           %% digital version of a document. Replace with `printed`
           %% to enable the default options for the printed version
           %% of a document.
  notable,   %% Causes the coloring of tables. Replace with `notable`
           %% to restore plain tables.
  nolof,     %% Prints the List of Figures. Replace with `nolof` to
           %% hide the List of Figures.
  nolot,     %% Prints the List of Tables. Replace with `nolot` to
           %% hide the List of Tables.
  %% More options are listed in the user guide at
  %% <http://mirrors.ctan.org/macros/latex/contrib/fithesis/guide/mu/fi.pdf>.
]{fithesis3}

%% The following section sets up the locales used in the thesis.
\usepackage[resetfonts]{cmap} %% cmap makes the PDF searchable and copyable
\usepackage[T1]{fontenc}
\usepackage[
  main=english, %% By using `czech` or `slovak` as the main locale
                %% instead of `english`, you can typeset the thesis
                %% in either Czech or Slovak, respectively.
  czech         %% The additional keys allow
]{babel}        %% foreign texts to be typeset as follows:
%%
%%   \begin{otherlanguage}{czech}   ... \end{otherlanguage}

%% The following section sets up the metadata of the thesis.
\thesissetup{
    date          = \the\year/\the\month/\the\day,
    university    = mu,
    faculty       = fi,
    type          = d,
    author        = Petr Velan,
    gender        = m,
    advisor       = {doc. Ing. Pavel Čeleda, Ph.D.},
    title         = {Next Generation Application-Aware Flow Monitoring},
    TeXtitle      = {Next Generation Application-Aware Flow Monitoring},
    keywords      = {network, monitoring, network flow, NetFlow, IPFIX},
    TeXkeywords   = {network, monitoring, network flow, NetFlow, IPFIX},
    bib           = bibliography.bib,
}
\thesislong{abstract}{
    This is the abstract of my thesis, which can

    span multiple paragraphs.


	    
Main contributions:
\begin{itemize}
	\item High-speed flow monitoring (100G, 2x80G)
	\item Addition of application layer information to flow monitoring
	\item Analysis of impact of traffic composition on application flow monitoring
	\item Proposal of new generation of application flow measurement - EventFlow 
\end{itemize}

Other contributions
\begin{itemize}
	\item Design of flexible flow collector for application flow processing
	\item Implementation of flexible IPFIX export for flowmon exporter
	\item Survey of methods for encrypted traffic classification
\end{itemize}

TODO: discuss privacy issues (where?)
}

\thesislong{thanks}{
    This is the acknowledgement for my thesis, which can

    span multiple paragraphs.
}

\usepackage{makeidx}      %% The `makeidx` package contains
\makeindex                %% helper commands for index typesetting.
%% These additional packages are used within the document:
\usepackage{paralist} %% Compact list environments
\usepackage{amsmath}  %% Mathematics
\usepackage{amsthm}
\usepackage{amsfonts}
\usepackage{url}      %% Hyperlinks
\usepackage{markdown} %% Lightweight Markup

% \usepackage{graphicx}
% \usepackage{a4wide}
% \usepackage{tabularx}
% \usepackage{amssymb}
% \usepackage[hyphens]{url}
% \usepackage[titletoc]{appendix}
% \usepackage{color}
% \usepackage{textcomp}
% \usepackage[hidelinks]{hyperref}
% \usepackage{pifont}
% \usepackage[table]{xcolor}
% \usepackage{subfig}
% \usepackage[ruled, lined, linesnumbered,norelsize]{algorithm2e}

\begin{document}

%------------------------------------------------------------------------------
% Intro
\chapter{Introduction}

% TODO: pouzit sablonu z hab prace

\begin{itemize}
	\item Motivation for the thesis
	\item Terminology, frequently used terms
	\item Research questions
	\item Thesis structure
\end{itemize}


% \section{Terminology}
% 
\section{Research Questions}

 
\section{Thesis Structure}



%------------------------------------------------------------------------------
% Background
\chapter{Thesis Background}
\label{chap:thesis-background}

This chapter describes the flow monitoring process from raw data acquisition to network anomaly detection.

% Use section 2.2 from thesis proposal

\section{Flow Definition}

\section{Flow Monitoring Architecture}

\subsection{Flow Exporter}

\subsection{Flow Collector}

%------------------------------------------------------------------------------
% State of the art
\chapter{State of the Art}
\label{chap:state-of-the-art}

This chapter should map the state of the art for the selected research questions.

\section{High-speed Packet Capture}
This section describes how the packets are captured on network and delivered to software for further processing.

Start with related work from IM2015 paper.

\section{Application Classification}
\subsection{Classification Using DPI/Signatures}
Start with 2.1.1 from thesis proposal. Describe existing approaches and tools. Performance limitations.

\subsection{Classification Using Flow-based Techniques}
Start with 2.4.1 from thesis proposal.
Cite only relevant surveys, there are too many works.
Nobody is concerned with performance.


\section{Extracting Information From Application Layer}

\section{High-speed Traffic Generation}
Existing approaches, problems.

\section{Open Issues}
Describe what are the open issues in this field.
\begin{itemize}
	\item Impact, limits and benefits of application flow monitoring
	\item Quality and availability of data sets
	\item Unified benchmarking environment
	\item Dealing with encrypted traffic
	\item Identifying user events in flows
\end{itemize}


\section{Summary}

% \section{Relevant Publications}


%------------------------------------------------------------------------------

\chapter{Flow Measurement Benchmarking Environment}
Describe an idea for benchmarking flow exporters or other applications that require application data.

\section{Network Traffic Characteristics}
How can we describe network traffic in general? How to differentiate between types of traffic for benchmarking purposes?

\section{Flow-preserving Traffic Generator}
Describe ideas behind new packet generator that is capable of generating traffic with real-world properties.

\section{Application Data Sets}
Describe data sets for the pfgen.

\section{Relevant Publications}
TODO: two publications, one for the data sets, one for the generator.

%------------------------------------------------------------------------------
% 
\chapter{Application Flow}

\section{Impacts of Application Flow}

\begin{itemize}
	\item What are the impacts of application protocol measurement on flow exporters?
\end{itemize}


\section{Application Flow Performance}

\begin{itemize}
	\item What are the limits of application protocol measurement on high-speed networks?
	\item 100\,G flow vs 10\,G application flow
\end{itemize}


\section{Application Flow Benefits}

\begin{itemize}
	\item How can application protocol information be used to improve flow measurement quality?
\end{itemize}


\section{Relevant Publications}

[1] Petr Velan, Tomáš Jirsík and Pavel Čeleda. Design and Evaluation of HTTP Protocol Parsers for IPFIX Measurement. In Lecture Notes in Computer Science, Vol. 8115, pages 136-147, Chemnitz, Germany, 2013.

TODO: NOMS 2016 (Experience session paper, August 2015): Paper on impact - CPU, memory impact. Large flow cache, maybe IPFIX stream compression. (Q2 2015)

IM2015

TODO: Journal paper Computer Networks (1.282 IF, check deadline): Similar paper as IM2015 on application measurement. Maybe together with paper on app measurement impact. (Q3 2015)

TODO: IMC 2015 (April): 100\,G flow measurement (need machine with bifurcation and good CPUs)

[3] Pavel Celeda, Petr Velan, Martin Rábek, Rick Hofstede and Aiko Pras. Large-Scale Geolocation for NetFlow. In IFIP/IEEE International Symposium on Integrated Network Management (IM 2013), pages 1015-1020, Ghent, Belgium, 2013.

[4] Martin Elich, Petr Velan, Tomáš Jirsík and Pavel Celeda. An Investigation Into Teredo and 6to4 Transition Mechanisms: Traffic Analysis. In 38th Annual IEEE Conference on Local Computer Networks (LCN 2013), pages 1046-1052, Sydney, Australia, 2013.

%------------------------------------------------------------------------------
% 
\chapter{Measurement of Encrypted Traffic}

\begin{itemize}
	\item What can be done about encrypted traffic visibility?
\end{itemize}

\section{Relevant Publications}
TODO: IJNM2014 survey (Q3 2014)

%------------------------------------------------------------------------------
% 
\chapter{Next Generation Flow}

\begin{itemize}
	\item How to add application level events to flow measurements? We want logical grouping of flow records based on user actions.
\end{itemize}

\section{Relevant Publications}
TODO: Use the thesis of Jakub Melichar to write a paper (Q1 2015)

%------------------------------------------------------------------------------
% Conclusion
\chapter{Conclusion}
\label{chap:conclusions}

\section{Answers to Research Questions}



\section{Further Research}


%------------------------------------------------------------------------------

% Print index
\makeatletter\thesis@blocks@clear\makeatother
\phantomsection %% Print the index and insert it into the
\addcontentsline{toc}{chapter}{\indexname} %% table of contents.
\printindex

\appendix %% Start the appendices.
% List of authored publications
\chapter{List of Authored Publications}

\section{Impacted Journals}

\section{Conference Proceedings}

\fullcite{1120462}

\section{Other Publications}

% Print bibliography at the end (should happen automatically,
% but this removes overleaf's warning)
\printbibliography[heading=bibintoc] %% Print the bibliography.

\end{document}
