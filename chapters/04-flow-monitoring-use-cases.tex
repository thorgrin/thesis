\chapter{Flow Monitoring Use Cases}

\begin{chapintro}

application layer can mean more information from tunnelled protocols or even IPv6.

% Martin Husák, Petr Velan, and Jan Vykopal. “Security Monitoring of HTTP Traffic Using Extended Flows”.
% Martin Elich, Petr Velan, Tomáš Jirsík, and Pavel Čeleda. “An Investigation Into Teredo and 6to4 Transition Mechanisms: Traffic Analysis”.
% Pavel Čeleda, Petr Velan, Martin Rábek, Rick Hofstede, and Aiko Pras. “Large-Scale Geolocation for NetFlow”
% Luuk Hendriks, Petr Velan, Ricardo de O. Schmidt, Pieter-Tjerk de Boer, and Aiko Pras. “Threats and Surprises Behind IPv6 Extension Headers”.
% Luuk Hendriks, Petr Velan, Ricardo de O. Schmidt, Pieter-Tjerk de Boer, and Aiko Pras. “Flow-Based Detection of IPv6-specific Network Layer Attacks”
The papers related to this chapter are~\cite{Husak-2015-Security, Elich-2013-Investigation, Celeda-2013-Large, Hendriks-2017-Flow, Hendriks-2017-Threats}.

The organisation of this chapter is as follows:
\begin{itemize}
%   \item Section~\ref{} 
%   \item Section~\ref{}
  \item Section~\ref{sec:use-cases-conclusions} concludes the chapter.
\end{itemize}

\end{chapintro}

\newpage

\section{Security Monitoring of HTTP Traffic Using Extended Flows}

\section{An Investigation Into Teredo and 6to4 Transition Mechanisms: Traffic Analysis}

\section{Threats and Surprises Behind IPv6 Extension Headers}

\section{Flow-Based Detection of IPv6-specific Network Layer Attacks}

\section{Large-Scale Geolocation for NetFlow}

\section{Conclusions}\label{sec:use-cases-conclusions}