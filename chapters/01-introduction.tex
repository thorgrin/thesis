\chapter{Introduction}

The concept of network flow monitoring was proposed in 1991 to facilitate accounting of network usage. Cisco's NetFlow became a de-facto standard for acquiring IP network and operational data in the following years. However, the main potential of flow monitoring became realised much later, in 2005, when Cisco engineers proposed to use NetFlow for anomaly detection and traffic analysis~\cite{CiscoSystems-2005-Cisco}. Flows records together with packet capture are the two main sources of data for intrusion detection systems nowadays. Moreover, the flow records are being used for data retention~\cite{Wanrooij-2005-Data}, which is mandatory for internet service providers in many countries. 

Due to the growing cybercrime industry and cyber espionage~\cite{CSIS-2013-Economic} the traffic analysis and security potential of flow monitoring have become more accentuated over time. To support traffic analysis, Cisco enriched its Flexible NetFlow~\cite{CiscoSystems-2008-Cisco} with information from the Network-Based Application Recognition (NBAR)~\cite{CiscoSystems--Network} in 2009. NBAR was initially used for QoS management on Cisco appliances. The trend of enriching flow records with information from application layer continued and resulted in the application-aware flow monitoring, which can be seen as a combination of Deep Packet Inspection (DPI) and flow monitoring. A deployment of flow monitoring has become standard practice since almost every enterprise networking equipment is able to export flow records nowadays. \citeauthor{Steinberger-2013-Anomaly} surveyed ISPs and network operators and found out that a majority of them uses flow monitoring for attack detection in their networks.

The importance of flow monitoring is growing. As the speed of network links increases, DPI-based intrusion detection systems are becoming incapable of handling the sheer amount of traffic. Moreover, increasing amount of encryption makes it harder still for DPI to be effective. The goal of this thesis is to advance flow monitoring techniques to achieve better application visibility and performance.

\section{Problem Statement}

When the flow monitoring is deployed on a network, the measured flow data are sent to a flow data processing system. The system facilitates flow analysis, reporting, and threat detection. All of these uses require high-quality flow data for their operation. For example, when some of the packets are not monitored or actively sampled, the quality of flow data is significantly reduced~\cite{Brauckhoff-2006-Impact}. Moreover, when the data contains artifacts~\cite{Hofstede-2013-Measurement}, the data analysis and threat detection can be impaired as well.

Maintaining high-quality flow monitoring system is a challenging task due to the constant changes in traffic structure and increasing volume~\cite{CiscoSystems-2017-Cisco}. We have identified three main topics that directly affect flow monitoring and flow data analysis. Firstly, the flow monitoring must keep pace with the increasing speed of networks. Therefore, the performance of flow monitoring must be studied and improved to match the speed of the network links. Secondly, as the network attacks are becoming more sophisticated, application layer information must be provided to enable more efficient threat detection. Lastly, the increasing amount of encrypted traffic makes application visibility difficult. Therefore, to maintain any degree of application visibility, novel approaches for monitoring of encrypted traffic must be explored. 

\subsection{Application Layer Information}

When we started our research in flow monitoring in 2012, application visibility in flow monitoring was a relatively new concept. With the exception of Flexible NetFlow utilising NBAR, the flow records contained only network and transport layer information. However, even the Flexible NetFlow provided only application recognition without extraction of any application-specific data. At the end of 2011 ntop released a version of their nProbe flow exporter capable of utilising OpenDPI library to provide information about application protocol~\cite{ntop-2011-Unveiling}. Still, the information provided was only application protocol name and identifier, which was very similar to what the Flexible NetFlow provided.

With the increasing number of discovered vulnerabilities in applications~\cite{Younan-2013-25}, we have found it essential to increase the capabilities of flow monitoring to detect more of these vulnerabilities. Therefore, we have decided to create true application-aware flow monitoring that would allow threat detection algorithms to utilise not only network and transport layer information, but also application layer information as well.

\subsection{Growing Network Speeds}

The implementation of flow monitoring started as an additional feature of routers and switches. However, as the main purpose of the networking devices is not flow monitoring, the quality of data is compromised under high load, where the networking capabilities are of higher importance than the monitoring itself. This, and the reason that the devices provided only limited configuration options lead to the development of flow monitoring on commodity hardware~\cite{Deri-2003-Passively}. The author showed that even monitoring of a Gigabit Ethernet on commodity hardware is a challenging problem. Therefore, even with increasing CPU frequency and the number of cores, flow monitoring of 10\,Gbps and faster networks requires the use of special techniques and optimisations. The recent advances in networking technologies lead to standardisation and deployment of 40\,Gbps and 100\,Gbps Ethernet links. Moreover, standards for 200\,Gpbs and 400\,Gbps Ethernet were approved at the end of 2017.

We have decided to research the possibilities of high-speed flow monitoring at these speeds. Moreover, since application-aware flow monitoring requires more performance than basic flow monitoring, we study the impact of processing of application payloads on the flow monitoring performance.
\subsection{Traffic Encryption}

Our decision to include application layer information in flows to increase network visibility and aid threat detection was based partially on the amount of unencrypted traffic that could be observed in the network. Research showed that only one-third of web pages could be browsed via HTTPS~\cite{Vratonjic-2013-Inconvenient} in 2013. However, the amount of encrypted traffic steadily increased, and more than 70 percent of web pages are loaded over HTTPS nowadays~\cite{ISRG-2018-Lets}. This massive change in the use of encryption necessitates the use of novel approaches to monitoring of the encrypted traffic. Therefore, a study of encrypted protocols, as well as an overview statistical methods of traffic classification, are needed to analyse the impact of encryption on the amount of information that can be provided by the flow monitoring.

% \section{Research Questions \& Approach} % Research Topics instead of research questions
\section{Research Goals}

The goals of this thesis are motivated by the discovered problems. We have identified the following list of goals:

\begin{itemize}
  \item Propose application flow monitoring which utilises application layer information to facilitate flow analysis and threat detection.
  \item Evaluate performance of flow monitoring and propose optimisations to facilitate monitoring of high-speed networks.
  \item Analyse options for monitoring of encrypted traffic, survey common encryption protocols and methods for encrypted traffic classification.
\end{itemize}


\section{Contributions}

In the course of pursuing the research goals, following contributions were made in the area of flow monitoring:

\begin{itemize}
    \item We have proposed a new definition of flow which respects the nature of flow monitoring. A formalisation of the flow definition is provided to make the definition more expressive. Moreover, we have shown how the definition can be used to describe the flow creation process formally as well. (Chapter~\ref{chap:network-flow-monitoring})
    \item We provide an overview of application flow monitoring that has been implemented in the past years. Consistent terminology is lacking in the area of application flow monitoring, which leads us to propose a definition of application flow monitoring together with appropriate terminology. (Chapter~\ref{chap:application-flow-monitoring})
    \item HTTP application flow monitoring has been implemented, and its performance evaluated. We show that evaluation of application monitoring performance depends on the used dataset even more heavily than basic flow monitoring. (Chapter~\ref{chap:application-flow-monitoring})
    \item The effect of additional information provided by application flow monitoring has been investigated on several different protocols. We have also analysed the feasibility of adding geolocation information to flow. (Chapter~\ref{chap:traffic-analysis-using-application-flow-monitoring})
    \item Flow monitoring performance of commodity hardware was analysed, and multiple optimisations were proposed. The use of hardware acceleration using FPGA-based network interface cards and possible optimisations with use of these cards were discussed as well. A monitoring system theoretically capable of achieving 160\,Gbps throughput was build based on the examined optimisations. (Chapter~\ref{chap:flow-monitoring-performance})
    \item Analysis of the most widely used encryption protocols was performed, and information suitable for use in application flow monitoring was discovered in the headers of these protocols. Traffic classification methods based on various statistical methods and machine learning were analysed. However, the usefulness of these methods for flow monitoring remains for further research. (Chapter~\ref{chap:measurement-of-encrypted-traffic})
    \item Three novel concepts that can benefit flow monitoring and especially application flow monitoring were presented. (Chapter~\ref{chap:next-generation-flow})
\end{itemize}

\section{Thesis Structure}

The rest of this thesis is structured as follows. Chapter~\ref{chap:network-flow-monitoring} describes flow monitoring, its history and present state as well as lessons learned from the deployment of flow monitoring system at CESNET National Research and Education Network. Chapter~\ref{chap:application-flow-monitoring} discusses application flow monitoring, its definition and terminology.  It also studies the implementation of application flow monitoring for HTTP protocol and evaluates its performance. Chapter~\ref{chap:traffic-analysis-using-application-flow-monitoring} consists of four use cases that highlight the benefits of application flow monitoring. Chapter~\ref{chap:flow-monitoring-performance} examines the performance of flow monitoring system on commodity hardware and a high-density monitoring system for high-speed networks is built and thoroughly evaluated. Chapter~\ref{chap:measurement-of-encrypted-traffic} analyses possible approaches to coping with monitoring of encrypted traffic. Chapter~\ref{chap:next-generation-flow} proposes three novel concepts for application flow monitoring. Finally, Chapter~\ref{chap:conclusions} concludes the thesis and presents directions for future work.
