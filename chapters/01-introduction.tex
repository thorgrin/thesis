\chapter{Introduction}

\itodo{
Motivation for flow monitoring: Rick’s thesis (data retention!) \\
Use introduction from Challenges of Application Flow Monitoring
}

\iimprove{Rewrite, the following is copied from COMSOC article}
A passive network traffic monitoring is an essential tool for network management, analysis, and security. There are several approaches to the passive network traffic monitoring which differ in the amount of reported information. On one hand, router interface counters can be reported using Simple Network Management Protocol to provide basic accounting information. On the other hand, deep packet inspection (DPI) provides detailed information about the content of the traffic and can be used for enforcing security policies or malware detection. The more detailed the analysis is, the more computing power it requires. Flow monitoring is a way of balancing the level of acquired information and necessary resources. Origins of flow monitoring date back to 1991 and the basic idea is to describe each network connection by a single record using the common properties of all packets in such a connection. Detailed information on the history and the current state of flow monitoring can be found in~\cite{Hofstede-2014-Flow}. Due to the level of information aggregation, the flow monitoring can achieve 100\,Gb/s throughput, and is, therefore, suitable for monitoring of large networks of internet service providers (ISP) and data centers.

Flow monitoring support is implemented in practically all enterprise routers and switches. It provides enough information to detect malicious or anomalous behavior and serves as a data source for many network security appliances. However, application-level attacks do not necessarily exhibit anomalous behavior on the network level and may remain undiscovered. As a DPI can discover these attacks more easily, flow monitoring is being enhanced with aspects of the DPI. The resulting method is called application flow monitoring (or measurement). Application flow records contain not only network level data but also information extracted from a payload of packets. Application flow monitoring can be seen as a compromise between the standard flow monitoring and full deep packet inspection: it balances performance and traffic visibility to provide as much information about the traffic as possible while being able to work on high-speed networks. The works of \citeauthor{Cejka-2015-Using} and \citeauthor{Husak-2015-Security}~\cite{Cejka-2015-Using, Husak-2015-Security} are examples of how application flow monitoring can be utilized to discover attacks on application protocol layer.

\itodo{
Anomaly detection and mitigation at internet scale: a survey\\
\url{http://dl.acm.org/citation.cfm?id=2525023.2525033}\\
\url{http://link.springer.com/chapter/10.1007/978-3-642-38998-6\_7}\\
Popisuje kdo všecko umí měřit flow, ipfix, ...
}

%%%%% intro from thesis proposal
Computer networks are an essential part of our world. They facilitate banking transactions, business and personal communication, shopping, education and many other aspects of modern life. Number of devices connected to the Internet is rapidly growing and is expected to grow even more in the future~\cite{number-of-attacks}. As the economic value of services provided over the Internet increases, it is becoming profitable for various crooks and attackers to deceive users and attack their devices~\cite{cybercrime-costs}.


Network traffic monitoring provides a scalable platform for building Intrusion Detection Systems (IDS). However, building such systems is always a compromise between accuracy and performance. To achieve the best accuracy, each packet must be dissected and classified. Such an approach is called Deep Packet Inspection (DPI) and there are several frameworks using this approach, such as Bro~\cite{Paxson:1999:BSD:337967.337972}, Suricata~\cite{suricata} and Snort~\cite{snort}. The DPI provides detailed information about contents of the packet, which is subsequently used to search for known malware signatures, exploits and other kinds of offensive traffic. When the amount of traffic that needs to be monitored exceeds capabilities of DPI systems, part of the traffic is not processed and the information is lost. 

IP flow monitoring can be seen as a complement of DPI, since it provides better performance and scalability for high-speed networks at a cost of lower accuracy. It was introduced as a feature of routers and switches that did not have much spare processing power. Each sequence of packets with same key characteristics is aggregated into one flow record (abbreviated simply as flow). Each flow record contains values of the key characteristics together with additional information about processed packets. Only packet headers must be processed to create IP flow record, which saves processing power significantly. Moreover, complete flow records can be evaluated at dedicated machine which allows for easy load distribution.

Cisco introduced flow monitoring in 1996~\cite{netflow-year}. Nowadays, it is widely used in academia, government and business. Flow records generated by Cisco devices are often used as data sources, although dedicated probes are used in more demanding environment with high-speed networks. It is possible to monitor 1\,Gbit link without any special equipment, but 10\,Gbit links require specialized drivers for passing packets for software processing~\cite{Iannaccone:2001:MVH:505202.505235}.

The IP flow monitoring is being used for even more use cases than it was ever intended for. Detection methods are leveraging various properties of flow records to extrapolate information that is not present in the flow records. As the demand for more accurate information increases, different vendors are starting to add more information from application layer to the flow records. This approach is called \emph{application flow} and Cisco Network Based Application Recognition (NBAR)~\cite{cisco-nbar} is an example of this approach.

The application flow monitoring needs more than just packet headers, it needs to examine the application data. Although this approach allows flow monitoring to gain some advantages of the DPI, it is at a cost of performance~\cite{1120462}. Therefore, the information gained from packets must be carefully selected, since an attempt to gain all available information would put the the solution on the same performance level as the DPI, or even below it.

Application flow monitoring is created as an attempt to provide application layer information using flow monitoring. However, impacts, benefits and disadvantages of application flow monitoring have not been studied in detail yet. We will contribute to this area in several ways. We will study the impact of application flow monitoring on flow exporters. Based on the results, we will propose improvements to the application flow monitoring that will help to cope with discovered disadvantages. We also believe that it is possible to utilize newly acquired application information to improve quality of the flows. Based on previous results, we will propose next generation flow measurement for application monitoring. The flows will match events in application protocol, e.g. web page download, instead of packet stream. This approach will allow IDS systems to apply more accurate techniques for detection intrusions and other malicious activities. Finally, to address the necessity of monitoring high-speed networks, we will investigate the performance of different approaches to application classification and application parsing with a computational complexity in mind.

\section{Problem Statement}


\section{Research Questions \& Approach} % Research Topics instead of research questions
% \section{Research Goals}

\begin{enumerate}
	\item How can flow creation process be formally described. [chap 2]
	\item How can flow monitoring benefit from application layer information. [chap 3]
	\item What is the performance impact of adding application layer information to flow monitoring. [chap 4]
	\item How to monitor high-speed networks (100G+), what can be done to accelerate (application) flow monitoring. [chap 4]
	\item Can application layer information be used to derive more information about the traffic (EventFlow). [chap 6]
	\item How can flow monitoring cope with increasing share of encrypted traffic. [chap 5]
\end{enumerate}


\section{Contributions}

% \itodo{
% Main contributions:\\
% - Flow definition\\
% - Addition of application layer information to flow monitoring\\
% - Analysis of impact of traffic composition on application flow monitoring\\
% - High-speed flow monitoring (100G, 2x80G)\\
% - Proposal of new generation of application flow measurement - EventFlow \\
% - Survey of methods for encrypted traffic classification\\
% 
% Other contributions\\
% - Design of flexible flow collector for application flow processing\\
% - Implementation of flexible IPFIX export for flowmon exporter\\
% 
% TODO: discuss privacy issues (where?)
% }

\section{Thesis Structure}
