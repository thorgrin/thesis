\chapter{Next Generation Flow}\label{chap:next-generation-flow}

\itodo{Popsat milniky flow monitoringu - pocatky, security, application visibility, high-speed, ...\\
Ukazat, kam se muze flow dostat dal - eventflow a metaflow\\
timeline obrazek}

\section{EventFlow}

\section{MetaFlow}
% zobecneni eventflow, vice urovni toku (zanorene protokoly), i aplikacni protokoly by mohly mit svoje specificke toky (kazdy Lx protokol svuj tok, vsecko provazane)
% PAM 2018

% dalsi moznosti se jevi exportovat toky a vedle nich aplikacni udalosti. nervat to primo do toku, protoze to cele ty toky rozbiji a neni to moc elegantni na zpracovani

\section{Flexible Flow Collector}

\itodo{
- IPFIXcol\\
- Collector design\\
- Problems with IPFIX protocol processing (Projít s Lukášem Hutákem)\\
- Flexible flow data storage (db vs nfdump vs FastBit vs new format?)\\
- IF IPFIX format needs to be described, look at Hofstede et al. V.E.\\

- Mediator: fix intro in chapter 2: section Flow Monitoring Architecture if described here

- Take a look at https://github.com/VerizonDigital/vflow
}

\section{Flexible Flow Data Storage}\label{sec:flexible-flow-data-storage}
\itodo{IM 2013 fbitdump vs nfdump comparison}

\section{Conclusions}

\section{Relevant Publications}

\itodo{
- AIMS 2012 ipfixcol design\\
- IM 2013 fbitdump vs nfdump comparison\\
- NOMS 2015 EventFlow
}